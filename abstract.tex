\chapter*{Abstract}
\addcontentsline{toc}{chapter}{Abstract}

\selectlanguage{german}

Proof-of-Stake (PoS) ist eine Klasse der Konsensus-Algorithmen, die in Bezug auf Stromverbrauch und Sicherheit signifikante Verbesserungen gegen"uber dem derzeit marktbeherrschend Proof-of-Work (PoW) Konsensus-Mechanismus bietet.


Der PoS-Konsensus-Algorithmus wird im Allgemeinen am besten wie folgt beschrieben: Ein Validierer ist ein Knoten im Netzwerk, der Transaktionen verifiziert und diese in From eines Blocks der Blockchain hinzuf"ugt. Jeder Knoten in dem System kann zu einem Validierer werden, indem eine bestimmte Anzahl Tokens der dazugeh"origen Kryptow"ahrung hinterlegt wird. Diese Tokens k"onnen nicht ausgegeben werden, solange der Validierer Teil des Validierersets ist. Diese Kaution wird verwendet, um b"osartige Knoten zu bestrafen. Das Recht f"ur das Hinzuf"ugen des n"achsten Blocks wird in einem dezentralisierten Auswahlverfahren gemacht und ist abh"angig von der Anzahl der hinterlegten Tokens jedes Validierers.


Diese Bachelorarbeit zeigt die Vorteile von PoS-Protokollen gegen"uber einem PoW-Konsensus-Mechanismus auf. Dabei wird auf den Stromverbrauch von Bitcoin eingegangen, der zur Zeit des Verfassens dieser Arbeit auf mehr als 5 Millionen USD pro Tag gesch"atzt wird. Ebenfalls wird auf das Risiko der Zentralisierung von Mining-Pools und Cloud-Mining-Platformen eingegangen und wieso die Eintrittsbarriere ins PoS-Validieren tiefer als in PoW-Mining ist. 

Weiter werden die Unterschiede zwischen den bereits existierenden PoS-Konsensus-Algorithmen aufgezeigt und analysiert. Aus diesen Analysen werden die allgemeinen Schwierigkeiten in PoS-Systemen dargelegt.

Bazo ist eine Kryptow"ahrung basierend auf der Blockchain Technologie, die in einer fr"uheren Bachelorarbeit an der Universit"at Z"urich entwickelt wurde. Es erlaubt die Erstellung von neuen Benutzerkonten und das Transferieren von Bazo Tokens zwischen Benutzerkonten. Aus Einfachheitsgr"unden hatte man sich damals f"ur einen PoW-Konsensus-Mechanismus entschieden. Das Ziel dieser Bachelorarbeit ist es, das Konsensus-Protokoll der Bazo Blockchain in ein PoS-System zu "uberf"uhren.


\selectlanguage{english}

In a previous bachelor thesis of the University of Zurich the Bazo cryptocurrency was developed. It supports the creation of new accounts and the transfer of funds on the blockchain. Due to simplicity Bazo uses a Proof-of-Work (PoW) consensus algorithm. The biggest disadvantages of a PoW consensus mechanism are the high energy consumption, the risk of centralization and the high entry barrier for new miners. Furthermore, in early stages of a new cryptocurrency a blockchain with a low number of miners poses a great risk of the 51\% attack. A successful implementation of a PoS-protocol could solve these problems mentioned above.

The goal of this bachelor thesis is to study and analyse current Proof-of-Stake (PoS) implementations, design a new system based on the findings and convert the Bazo consensus mechanism from PoW to PoS.
