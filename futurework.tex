\chapter{Future Work}

The consensus mechanism employed in Bazo is unique and does not exist in any other blockchain. Thus, the security of the newly designed protocol must be thoroughly tested. 

Although the random validator election process works similar as in PPCoin's protocol, Bazo claims to be less vulnerable towards stake grinding attacks. In a future project the Bazo blockchain should be tested if this claim is justified by attempting to perform such an attack. 

Furthermore, a mathematically underpinned analysis is needed in order to define the system parameters securely. For now, the minimum-waiting-time parameter (see Section \ref{system-parameters}) can only be adjusted with a ConfigTx signed by a root account. However, this parameter is of great importance in regard to security (as shown in Section \ref{stake-grinding-attacks}). This poses a risk of a security flaw if the responsible root accounts do not adjust this parameter when topology of the validator set changes. A dynamic analysis of the stake distribution among all validators is desired, which also updates the parameter for the minimum-waiting-time automatically within the system. 

The same applies for the slashing-window-size and the accepted-time-difference. For the accepted-time-difference the disparity of the clock speeds on different machines running the Bazo protocol have to be investigated.